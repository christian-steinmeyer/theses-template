\section{Section 0}\label{sec:zero}

This is a reference \cite{tur38}. This is an acronym: \ac{MI}. Fun fact: when using it again, it will only be displayed like such: \ac{MI}.

Note, that the gray boxes on the cover page can be replaced. Simply replace the \code{logo.png} file in the \code{images} folder.

cref Demonstration: Cref at beginning of sentence, cref in all other cases. \Cref{fig:logo} shows a simple fact, although \cref{fig:logo} could also show something else. \Cref{tab:simple} shows a simple fact, although \cref{tab:simple} could also show something else. \Cref{sec:one} shows a simple fact, although \cref{sec:one} could also show something else.

\image{logo}{Simple Figure}

Brackets work as designed: <test>

\begin{inparaenum}
\item All these items...
\item ...appear in one line
\item This is enabled by the paralist package.
\end{inparaenum}

\javafile{SetOperation}{A simple Javafile as an example}

